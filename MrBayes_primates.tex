\documentclass{article}\usepackage[]{graphicx}\usepackage[]{color}
%% maxwidth is the original width if it is less than linewidth
%% otherwise use linewidth (to make sure the graphics do not exceed the margin)
\makeatletter
\def\maxwidth{ %
  \ifdim\Gin@nat@width>\linewidth
    \linewidth
  \else
    \Gin@nat@width
  \fi
}
\makeatother

\definecolor{fgcolor}{rgb}{0.345, 0.345, 0.345}
\newcommand{\hlnum}[1]{\textcolor[rgb]{0.686,0.059,0.569}{#1}}%
\newcommand{\hlstr}[1]{\textcolor[rgb]{0.192,0.494,0.8}{#1}}%
\newcommand{\hlcom}[1]{\textcolor[rgb]{0.678,0.584,0.686}{\textit{#1}}}%
\newcommand{\hlopt}[1]{\textcolor[rgb]{0,0,0}{#1}}%
\newcommand{\hlstd}[1]{\textcolor[rgb]{0.345,0.345,0.345}{#1}}%
\newcommand{\hlkwa}[1]{\textcolor[rgb]{0.161,0.373,0.58}{\textbf{#1}}}%
\newcommand{\hlkwb}[1]{\textcolor[rgb]{0.69,0.353,0.396}{#1}}%
\newcommand{\hlkwc}[1]{\textcolor[rgb]{0.333,0.667,0.333}{#1}}%
\newcommand{\hlkwd}[1]{\textcolor[rgb]{0.737,0.353,0.396}{\textbf{#1}}}%

\usepackage{framed}
\makeatletter
\newenvironment{kframe}{%
 \def\at@end@of@kframe{}%
 \ifinner\ifhmode%
  \def\at@end@of@kframe{\end{minipage}}%
  \begin{minipage}{\columnwidth}%
 \fi\fi%
 \def\FrameCommand##1{\hskip\@totalleftmargin \hskip-\fboxsep
 \colorbox{shadecolor}{##1}\hskip-\fboxsep
     % There is no \\@totalrightmargin, so:
     \hskip-\linewidth \hskip-\@totalleftmargin \hskip\columnwidth}%
 \MakeFramed {\advance\hsize-\width
   \@totalleftmargin\z@ \linewidth\hsize
   \@setminipage}}%
 {\par\unskip\endMakeFramed%
 \at@end@of@kframe}
\makeatother

\definecolor{shadecolor}{rgb}{.97, .97, .97}
\definecolor{messagecolor}{rgb}{0, 0, 0}
\definecolor{warningcolor}{rgb}{1, 0, 1}
\definecolor{errorcolor}{rgb}{1, 0, 0}
\newenvironment{knitrout}{}{} % an empty environment to be redefined in TeX

\usepackage{alltt}
\usepackage[sc]{mathpazo}
\usepackage[T1]{fontenc}
\usepackage{geometry}
\usepackage{natbib}
\usepackage[utf8]{inputenc}
\geometry{verbose,tmargin=2.5cm,bmargin=2.5cm,lmargin=2.5cm,rmargin=2.5cm}
\setcounter{secnumdepth}{2}
\setcounter{tocdepth}{2}
\usepackage{url}
\usepackage[unicode=true,pdfusetitle,
 bookmarks=true,bookmarksnumbered=true,bookmarksopen=true,bookmarksopenlevel=2,
 breaklinks=false,pdfborder={0 0 1},backref=false,colorlinks=false]
 {hyperref}
\hypersetup{
 pdfstartview={XYZ null null 1}}
\IfFileExists{upquote.sty}{\usepackage{upquote}}{}
\begin{document}



\title{MrBayes practical (with primates!)}

\author{Thomas Guillerme\\\href{mailto:t.guillerme@imperial.ac.uk}{t.guillerme@imperial.ac.uk}}

\maketitle

\tableofcontents

\section{Installing MrBayes}
You can download \textbf{MrBayes} \citep{Ronquist2012mrbayes} for any platform (\href{http://mrbayes.sourceforge.net/}{here}).
A simple tutorial is available (\href{http://mrbayes.sourceforge.net/wiki/index.php/Tutorial_3.2}{here}) along with a manual (\href{http://mrbayes.sourceforge.net/wiki/index.php/Manual_3.2}{here}).

\section{The data}
For this practical we are going two files, one matrix containing phylogenetic information (\texttt{NEXUS} format; \texttt{.nex}) from \cite{hayasaka1988molecular} ad \cite{stevens2013palaeontological} and a second file being a MrBayes executable containing the phylogenetic information and MrBayes instructions:
\begin{itemize}
\item{\texttt{TotalEvidence\_Primates.nex}}: alignment containing 898 molecular characters and 191 morphological ones for 20 taxa.
\item{\texttt{TotalEvidence\_Primates\_exe.nex}}: the same alignment but with more detailed MrBayes instructions.
\end{itemize}

\section{Starting MrBayes}
To start MrBayes, simply double click on the executable or, for terminal users, enter \texttt{mb} in the terminal.
Then start running a log file to save all your analysis by using the \texttt{log} command:

\begin{knitrout}
\definecolor{shadecolor}{rgb}{0.969, 0.969, 0.969}\color{fgcolor}\begin{kframe}
\begin{alltt}
\hlkwd{log} start filename=\hlstr{MrBayes_practical.txt}
\end{alltt}
\end{kframe}
\end{knitrout}

\section{Running the analysis}
\subsection{Reading the data}
We can read the data into MrBayes by using the command \texttt{execute} (make sure you're in the right working directory!).
\begin{knitrout}
\definecolor{shadecolor}{rgb}{0.969, 0.969, 0.969}\color{fgcolor}\begin{kframe}
\begin{alltt}
\hlkwd{execute} \hlstr{TotalEvidence\_Primates.nex}
\end{alltt}
\end{kframe}
\end{knitrout}

We can now manually manipulate some characteristic of our data file, for example by setting an outgroup:

\begin{knitrout}
\definecolor{shadecolor}{rgb}{0.969, 0.969, 0.969}\color{fgcolor}\begin{kframe}
\begin{alltt}
\hlkwd{outgroup} \hlstr{Lemur_catta}
\end{alltt}
\end{kframe}
\end{knitrout}

\noindent Or by creating the group that contains only the fossil taxa (here we define the taxa number 13 to 20 to be fossils).

\begin{knitrout}
\definecolor{shadecolor}{rgb}{0.969, 0.969, 0.969}\color{fgcolor}\begin{kframe}
\begin{alltt}
\hlkwd{taxset} \hlstr{fossils}=13-20
\end{alltt}
\end{kframe}
\end{knitrout}

\noindent And some topological constraints to make the analysis a bit faster and to allow to calibrate these groups in the node calibration analysis.
\begin{knitrout}
\definecolor{shadecolor}{rgb}{0.969, 0.969, 0.969}\color{fgcolor}\begin{kframe}
\begin{alltt}
\hlkwd{constraint} \hlstr{root}=\hlstr{1-.}
\hlkwd{constraint} \hlstr{Haplorhini}=\hlstr{2-.}
\hlkwd{constraint} \hlstr{Simiiformes}=\hlstr{3-.}
\hlkwd{constraint} \hlstr{Catarrhini}=\hlstr{3-11}
\hlkwd{constraint} \hlstr{Hominidae}=\hlstr{3-7}
\hlkwd{constraint} \hlstr{Cercopithecidae}=\hlstr{8-11}
\end{alltt}
\end{kframe}
\end{knitrout}

\subsection{Setting the data partitions}
We can then partition our data by using \texttt{charset}.
We need a DNA and a morphology partition.

\begin{knitrout}
\definecolor{shadecolor}{rgb}{0.969, 0.969, 0.969}\color{fgcolor}\begin{kframe}
\begin{alltt}
\hlkwd{charset} \hlstr{DNA}=\hlstr{1-898}
\hlkwd{charset} \hlstr{morphology}=\hlstr{899-1089}
\end{alltt}
\end{kframe}
\end{knitrout}

\indent Now that the partitions are identified we need to specify that we want to use both partitions.
\begin{knitrout}
\definecolor{shadecolor}{rgb}{0.969, 0.969, 0.969}\color{fgcolor}\begin{kframe}
\begin{alltt}
\hlkwd{partition} \hlstr{my_parition}=\hlstr{2}: \hlstr{DNA}, \hlstr{morphology}
set \hlkwd{partition}=\hlstr{my_parition}
\end{alltt}
\end{kframe}
\end{knitrout}

\subsection{Setting the evolutionary models}
Now that our data are ready to use, we have to parametrise our phylogenetic model.
We need to set up a substitution model for each partition and (later on) a clock model for dating our phylogeny.


For the molecular data, we can go with a \textbf{GTR} model (General Time Reversible).
This model allows each transition rate between each nucleotide to be different (see \href{https://en.wikipedia.org/wiki/Models_of_DNA_evolution}{here for more models and more (easy) explanations}).
To simplify the model, we can also set a finite number of transition rates categories (for example, really slow, slow, medium, fast, etc.) that will follow a certain distribution.
Here we are going to use the \textbf{Gamma} ($\Gamma$) distribution with 4 distinct categories.

\subsubsection*{Model for the molecular data (GTR + 4$\Gamma$)}
\begin{knitrout}
\definecolor{shadecolor}{rgb}{0.969, 0.969, 0.969}\color{fgcolor}\begin{kframe}
\begin{alltt}
\hlkwd{lset} applyto=\hlstr{(1)} nucmodel=\hlstr{4by4} nst=\hlstr{6} rates=\hlstr{gamma} Ngammacat=\hlstr{4} covarion=\hlstr{no}
\end{alltt}
\end{kframe}
\end{knitrout}

\noindent For the morphological data, we are going to use the M\textit{k} model that assumes an equal transition rate between each state for each character.
Again, to simplify the model, we will set a 4 transition rates categories following a Gamma distribution.

\subsubsection*{Model for the molecular data (M\textit{k} + 4$\Gamma$)}
\begin{knitrout}
\definecolor{shadecolor}{rgb}{0.969, 0.969, 0.969}\color{fgcolor}\begin{kframe}
\begin{alltt}
\hlkwd{lset} applyto=\hlstr{(2)} nst=\hlstr{1} rates=\hlstr{gamma} Ngammacat=\hlstr{4}
\end{alltt}
\end{kframe}
\end{knitrout}

\noindent We can look at our model settings at any time by typing the following:
\begin{knitrout}
\definecolor{shadecolor}{rgb}{0.969, 0.969, 0.969}\color{fgcolor}\begin{kframe}
\begin{alltt}
help \hlkwd{lset}
\end{alltt}
\end{kframe}
\end{knitrout}


\subsection{Setting the priors}
We can also set some Bayesian priors for the analysis.
Most of the priors can be set to be flat (i.e. uninformative: the information is only a prior in a Bayesian sense and is not giving an \textit{a priori} to our analysis).
\begin{knitrout}
\definecolor{shadecolor}{rgb}{0.969, 0.969, 0.969}\color{fgcolor}\begin{kframe}
\begin{alltt}
help \hlkwd{prset}
\end{alltt}
\end{kframe}
\end{knitrout}

\noindent We are just going to change the rate priors to be variable.
\begin{knitrout}
\definecolor{shadecolor}{rgb}{0.969, 0.969, 0.969}\color{fgcolor}\begin{kframe}
\begin{alltt}
\hlkwd{prset} applyto=\hlstr{(all)} ratepr=\hlstr{variable}
\end{alltt}
\end{kframe}
\end{knitrout}

\subsection{Summarising the parameters}
Finally, we can summarise our whole model parameters by typing the following:
\begin{knitrout}
\definecolor{shadecolor}{rgb}{0.969, 0.969, 0.969}\color{fgcolor}\begin{kframe}
\begin{alltt}
\hlkwd{showmodel}
\end{alltt}
\end{kframe}
\end{knitrout}

Because all these manipulations only concern the data set, it might be a good idea to save them directly in the \texttt{TotalEvidence\_Primates.nex} file so that we don't have to retype them each time prior to any dating analysis.
This can be done between the following tags at the end of your nexus document :

\begin{knitrout}
\definecolor{shadecolor}{rgb}{0.969, 0.969, 0.969}\color{fgcolor}\begin{kframe}
\begin{alltt}
begin mrbayes;
    ...
end;
\end{alltt}
\end{kframe}
\end{knitrout}

\noindent Using the indentation is a good practice for emphasizing that the \texttt{...} will be executed within the \texttt{begin mrbayes} tag.
You can add all the commands below in place of the \texttt{...} and \textbf{finish all command lines by a semi-colon (\texttt{;})}.
You can also use the square brackets (\texttt{[]}) to write down some comments, they will be ignored by MrBayes.

\subsection{The tip-dated analysis}
\subsubsection{Set up the tip ages}
The first step in a Tip-dating analysis is to set up the age of the tips.
We can only inform the fossils ages since the living species have an aged assumed to be 0 million years old by default.
\begin{knitrout}
\definecolor{shadecolor}{rgb}{0.969, 0.969, 0.969}\color{fgcolor}\begin{kframe}
\begin{alltt}
\hlkwd{calibrate} \hlstr{Catopithecus_browni}=\hlstr{Fixed(36)}
\hlkwd{calibrate} \hlstr{Aegyptopithecus_zeuxis}=\hlstr{Fixed(33)}
\hlkwd{calibrate} \hlstr{Dendropithecus_macinnesi}=\hlstr{Fixed(20)}
\hlkwd{calibrate} \hlstr{Noropithecus_bulukensis}=\hlstr{Fixed(18)}
\hlkwd{calibrate} \hlstr{Victoriapithecus_macinnesi}=\hlstr{Fixed(15)}
\hlkwd{calibrate} \hlstr{Proconsul_major}=\hlstr{Fixed(20)}
\hlkwd{calibrate} \hlstr{Afropithecus_turkanensis}=\hlstr{Fixed(16)}
\hlkwd{calibrate} \hlstr{Morotopithecus_bishopi}=\hlstr{Fixed(18)}
\end{alltt}
\end{kframe}
\end{knitrout}

\noindent We can also set up some calibrations to fix the age of the root and the Haplorhini group.
Because we are not sure of the age of fossils, we can use a simple uniform distribution between the oldest and the youngest age estimate.
\begin{knitrout}
\definecolor{shadecolor}{rgb}{0.969, 0.969, 0.969}\color{fgcolor}\begin{kframe}
\begin{alltt}
\hlkwd{calibrate} \hlstr{root}=\hlstr{uniform(60.99,76.72)}
\hlkwd{calibrate} \hlstr{Haplorhini}=\hlstr{uniform(57.62,69.59)}
\end{alltt}
\end{kframe}
\end{knitrout}

\subsubsection{Set up the clock model}
Now we can set up the clock model by specifying that the branch length should represent time which is uniformly distributed along the branches (i.e. one unit of branch length = one unit of time across the whole tree; \texttt{brlenspr=clock:uniform}).
We can then set the clock to be relaxed by using the \textbf{IGR} (Independent Gamma rates) model for the clock rate variation.
This model assumes an independent rate on each branch distributed following a Gamma distribution (\texttt{clockvarpr=igr}).
Finally we can set some node calibration using the calibration we set up before.

\begin{knitrout}
\definecolor{shadecolor}{rgb}{0.969, 0.969, 0.969}\color{fgcolor}\begin{kframe}
\begin{alltt}
\hlkwd{prset} brlenspr=\hlstr{clock:uniform}
\hlkwd{prset} clockvarpr=\hlstr{igr}
\hlkwd{prset} nodeagepr=\hlstr{calibrated}
\hlkwd{prset} clockratepr=\hlstr{normal(0.01,0.005)}
\end{alltt}
\end{kframe}
\end{knitrout}
\noindent Along with our topological constraint:
\begin{knitrout}
\definecolor{shadecolor}{rgb}{0.969, 0.969, 0.969}\color{fgcolor}\begin{kframe}
\begin{alltt}
\hlkwd{prset} topologypr=\hlstr{constraints(root, Haplorhini)}
\end{alltt}
\end{kframe}
\end{knitrout}

\subsubsection{Run the analysis}
Finally we can run the analysis by setting up some of the MCMC parameters: the number of chains (\texttt{nchains=4}), the number of independent runs (\texttt{nruns=2}), which generations to sample (\texttt{samplefreq=1000}), which ones to print (\texttt{printfr=100}) and when to run the convergence diagnosis between both chains (\texttt{diagnfreq=2500}).
We can finally run the MCMC for a certain number of generations (1500000!) and see if the two chains converge.
\begin{knitrout}
\definecolor{shadecolor}{rgb}{0.969, 0.969, 0.969}\color{fgcolor}\begin{kframe}
\begin{alltt}
\hlkwd{mcmcp} temp=\hlstr{0.1} nchain=\hlstr{4} samplefreq=\hlstr{1000} printfr=\hlstr{100} nruns=\hlstr{2} diagnfreq=\hlstr{2500}
\hlkwd{mcmcp} filename=\hlstr{Primates-tip_dating}
\hlkwd{mcmc} ngen=\hlstr{150000} Stoprule=\hlstr{YES} stopval=\hlstr{0.01}
\end{alltt}
\end{kframe}
\end{knitrout}

\subsubsection{Save the tree}
Ones the analysis is over, we can save the parameters estimations by typing:
\begin{knitrout}
\definecolor{shadecolor}{rgb}{0.969, 0.969, 0.969}\color{fgcolor}\begin{kframe}
\begin{alltt}
\hlkwd{sump} filename=\hlstr{Primates-tip_dating} relburnin=\hlstr{YES} burninfrac=\hlstr{0.25}
\end{alltt}
\end{kframe}
\end{knitrout}

\noindent Each parameters are saved in the \texttt{Primates-tip\_dating.pstat} file.

\noindent As well as the tree by typing:
\begin{knitrout}
\definecolor{shadecolor}{rgb}{0.969, 0.969, 0.969}\color{fgcolor}\begin{kframe}
\begin{alltt}
\hlkwd{sumt} filename=\hlstr{Primates-tip_dating} relburnin=\hlstr{YES} burninfrac=\hlstr{0.25}
\end{alltt}
\end{kframe}
\end{knitrout}

\noindent The consensus tree is saved in the \texttt{Primates-tip\_dating.con.tre} file.

\subsection{The calibrated analysis}
Secondly we want to compare our results to the node calibration analysis (without fossils).
Most parameters are the same so they will not be described in details.

\subsubsection{Remove the fossil species}
We first have to remove our fossil partition:
\begin{knitrout}
\definecolor{shadecolor}{rgb}{0.969, 0.969, 0.969}\color{fgcolor}\begin{kframe}
\begin{alltt}
\hlkwd{delete} \hlstr{fossils}
\end{alltt}
\end{kframe}
\end{knitrout}

\subsubsection{Set up the calibrations}
We then need to set up our calibration (with uniform distributions for accounting for uncertainty).
\begin{knitrout}
\definecolor{shadecolor}{rgb}{0.969, 0.969, 0.969}\color{fgcolor}\begin{kframe}
\begin{alltt}
\hlkwd{calibrate} \hlstr{root}=\hlstr{uniform(60.99,76.72)}
\hlkwd{calibrate} \hlstr{Haplorhini}=\hlstr{uniform(57.62,69.59)}
\hlkwd{calibrate} \hlstr{Simiiformes}=\hlstr{uniform(33.55,49.48)}
\hlkwd{calibrate} \hlstr{Catarrhini}=\hlstr{uniform(57.62,69.59)}
\hlkwd{calibrate} \hlstr{Hominidae}=\hlstr{uniform(11.04,20.8)}
\hlkwd{calibrate} \hlstr{Cercopithecidae}=\hlstr{uniform(8.93,18.27)}
\end{alltt}
\end{kframe}
\end{knitrout}

\subsubsection{Set up the clock model}
We can use the same parameters as for the tip-dating analysis.
\begin{knitrout}
\definecolor{shadecolor}{rgb}{0.969, 0.969, 0.969}\color{fgcolor}\begin{kframe}
\begin{alltt}
\hlkwd{prset} brlenspr=\hlstr{clock:uniform}
\hlkwd{prset} clockvarpr=\hlstr{igr}
\hlkwd{prset} nodeagepr=\hlstr{calibrated}
\hlkwd{prset} clockratepr=\hlstr{normal(0.01,0.005)}
\end{alltt}
\end{kframe}
\end{knitrout}

\noindent Along with our topological constraint:
\begin{knitrout}
\definecolor{shadecolor}{rgb}{0.969, 0.969, 0.969}\color{fgcolor}\begin{kframe}
\begin{alltt}
\hlkwd{prset} topologypr=\hlstr{constraints(root, Haplorhini)}
\end{alltt}
\end{kframe}
\end{knitrout}
%\hlkwd{prset} topologypr=\hlstr{constraints(root, Haplorhini, Simiiformes, Catarrhini, Hominidae, Cercopithecidae)}
%\hlkwd{prset} topologypr=\hlstr{constraints(root, Haplorhini)}


\subsubsection{Run the analysis}
\begin{knitrout}
\definecolor{shadecolor}{rgb}{0.969, 0.969, 0.969}\color{fgcolor}\begin{kframe}
\begin{alltt}
\hlkwd{mcmcp} temp=\hlstr{0.1} nchain=\hlstr{4} samplefreq=\hlstr{1000} printfr=\hlstr{100} nruns=\hlstr{2} diagnfreq=\hlstr{2500}
\hlkwd{mcmcp} filename=\hlstr{Primates-node_dating}
\hlkwd{mcmc} ngen=\hlstr{150000} Stoprule=\hlstr{YES} stopval=\hlstr{0.01}
\end{alltt}
\end{kframe}
\end{knitrout}

\subsubsection{Save the tree}

Summarising the parameters:
\begin{knitrout}
\definecolor{shadecolor}{rgb}{0.969, 0.969, 0.969}\color{fgcolor}\begin{kframe}
\begin{alltt}
\hlkwd{sump} filename=\hlstr{Primates-node_dating} relburnin=\hlstr{YES} burninfrac=\hlstr{0.25}
\end{alltt}
\end{kframe}
\end{knitrout}

\noindent Summarising the trees:
\begin{knitrout}
\definecolor{shadecolor}{rgb}{0.969, 0.969, 0.969}\color{fgcolor}\begin{kframe}
\begin{alltt}
\hlkwd{sumt} filename=\hlstr{Primates-node_dating} relburnin=\hlstr{YES} burninfrac=\hlstr{0.25}
\end{alltt}
\end{kframe}
\end{knitrout}

\section{Comparing both trees}
We can now use \texttt{R} to compare both trees and see what is the effect of the both methods.

\subsection{Load the packages}
\begin{knitrout}
\definecolor{shadecolor}{rgb}{0.969, 0.969, 0.969}\color{fgcolor}\begin{kframe}
\begin{alltt}
Theses are the pacakges we will be using.
\hlcom{## Installing the packages}
\hlkwd{if}(!\hlkwd{require}(ape)) \hlkwd{install.packages}(\hlstr{"ape"})
\hlkwd{if}(!\hlkwd{require}(devtools)) \hlkwd{install.packages}(\hlstr{"devtools"})
\hlkwd{library}(ape, devtools)
\hlkwd{install_github}(\hlstr{"TGuillerme/dispRity"}, ref = \hlstr{"release"})
\end{alltt}
\end{kframe}
\end{knitrout}

\subsection{Import the trees}
First we need to import the trees in \texttt{R}.
Note that here I am using two pre-made trees, the ideal would be to use your own trees!
\begin{knitrout}
\definecolor{shadecolor}{rgb}{0.969, 0.969, 0.969}\color{fgcolor}\begin{kframe}
\begin{alltt}
\hlcom{## Reading the trees}
\hlcom{## WARNING: Make sure you're in the right folder! Use setwd(), getwd() and}
\hlcom{## list.files() to navigate.}
\hlstd{node_calibrated_tree} \hlkwb{<-} \hlkwd{read.nexus}\hlstd{(}\hlstr{"Primates-node_dating.bkp.tre"}\hlstd{)}
\hlstd{tip_dated_tree}       \hlkwb{<-} \hlkwd{read.nexus}\hlstd{(}\hlstr{"Primates-tip_dating.bkp.tre"}\hlstd{)}

\hlcom{## Removing the fossils from the tip dated tree}
\hlstd{tip_dated_tree_living} \hlkwb{<-} \hlkwd{drop.tip}\hlstd{(tip_dated_tree,} \hlkwc{tip} \hlstd{=} \hlkwd{c}\hlstd{(}\hlstr{"Catopithecus_browni"}\hlstd{,}
  \hlstr{"Aegyptopithecus_zeuxis"}\hlstd{,} \hlstr{"Dendropithecus_macinnesi"}\hlstd{,}
  \hlstr{"Noropithecus_bulukensis"}\hlstd{,} \hlstr{"Victoriapithecus_macinnesi"}\hlstd{,} \hlstr{"Proconsul_major"}\hlstd{,}
  \hlstr{"Afropithecus_turkanensis"}\hlstd{,} \hlstr{"Morotopithecus_bishopi"}\hlstd{))}
\end{alltt}
\end{kframe}
\end{knitrout}

\subsection{Comparing the trees}
Then we can compare the branch length and the topology from each tree:

\begin{knitrout}
\definecolor{shadecolor}{rgb}{0.969, 0.969, 0.969}\color{fgcolor}\begin{kframe}
\begin{alltt}
\hlcom{## Calculate the age difference between both trees for each nodes}
\hlkwd{library}\hlstd{(dispRity)}
\hlstd{node_calibrated_ages}  \hlkwb{<-} \hlkwd{tree.age}\hlstd{(node_calibrated_tree)}
\hlstd{tip_dated_living_ages} \hlkwb{<-} \hlkwd{tree.age}\hlstd{(tip_dated_tree_living)}
\hlstd{tip_dated_ages}        \hlkwb{<-} \hlkwd{tree.age}\hlstd{(tip_dated_tree)}
\end{alltt}
\end{kframe}
\end{knitrout}

\noindent We can now plot the two trees (including the two versions of the tip dated tree):

\begin{knitrout}
\definecolor{shadecolor}{rgb}{0.969, 0.969, 0.969}\color{fgcolor}\begin{kframe}
\begin{alltt}
\hlcom{## Setting the graphical parameters}
\hlstd{op} \hlkwb{<-} \hlkwd{par}\hlstd{(}\hlkwc{mfrow} \hlstd{= (}\hlkwd{c}\hlstd{(}\hlnum{1}\hlstd{,}\hlnum{3}\hlstd{)))}

\hlcom{## Plotting the node calibrated tree}
\hlkwd{plot}\hlstd{(node_calibrated_tree,} \hlkwc{main} \hlstd{=} \hlstr{"node calibrated"}\hlstd{,} \hlkwc{cex}\hlstd{=}\hlnum{0.7}\hlstd{)}
\hlcom{## Adding the age of the nodes calculates using tree.age()}
\hlkwd{nodelabels}\hlstd{(}\hlkwc{text} \hlstd{=} \hlkwd{round}\hlstd{(node_calibrated_ages}\hlopt{$}\hlstd{ages[}\hlnum{13}\hlopt{:}\hlnum{23}\hlstd{],} \hlnum{2}\hlstd{),} \hlkwc{cex} \hlstd{=} \hlnum{0.5}\hlstd{)}
\hlcom{## Adding the time ruler}
\hlkwd{axisPhylo}\hlstd{()}

\hlcom{## Plotting the tip dated tree without fossils (with the labels facing the}
\hlcom{## calibrated one)}
\hlkwd{plot}\hlstd{(tip_dated_tree_living,} \hlkwc{main} \hlstd{=} \hlstr{"tip-dated (living only)"}\hlstd{,} \hlkwc{direction} \hlstd{=} \hlstr{"l"}\hlstd{,}
  \hlkwc{cex}\hlstd{=}\hlnum{0.7}\hlstd{)}
\hlcom{## Adding the age of the nodes calculates using tree.age()}
\hlkwd{nodelabels}\hlstd{(}\hlkwc{text} \hlstd{=} \hlkwd{round}\hlstd{(tip_dated_living_ages}\hlopt{$}\hlstd{ages[}\hlnum{13}\hlopt{:}\hlnum{23}\hlstd{],} \hlnum{2}\hlstd{),} \hlkwc{cex} \hlstd{=} \hlnum{0.5}\hlstd{)}
\hlcom{## Adding the time ruler}
\hlkwd{axisPhylo}\hlstd{()}

\hlcom{## Plotting the tip dated tree with fossils}
\hlkwd{plot}\hlstd{(tip_dated_tree,} \hlkwc{main} \hlstd{=} \hlstr{"tip-dated"}\hlstd{,} \hlkwc{cex}\hlstd{=}\hlnum{0.7}\hlstd{)}
\hlcom{## Adding the age of the nodes calculates using tree.age()}
\hlkwd{nodelabels}\hlstd{(}\hlkwc{text} \hlstd{=} \hlkwd{round}\hlstd{(tip_dated_ages}\hlopt{$}\hlstd{ages[}\hlnum{21}\hlopt{:}\hlnum{38}\hlstd{],} \hlnum{2}\hlstd{),} \hlkwc{cex} \hlstd{=} \hlnum{0.5}\hlstd{)}
\hlcom{## Adding the time ruler}
\hlkwd{axisPhylo}\hlstd{()}

\hlcom{## Resetting the graphical parameters to default}
\hlkwd{par}\hlstd{(op)}
\end{alltt}
\end{kframe}
\end{knitrout}

\section{Questions}
\begin{enumerate}
\item What are the main differences between the trees? Are their also differences in parameters estimations?
\item Which one of these trees is the best? For which purpose is it better suited?
\item Can you see some caveats for each method?
\item Which method took longer and why?
\end{enumerate}


\bibliographystyle{sysbio}
\bibliography{References}

\end{document}
\end{document}



% Node dating ~ 145000 gen for short 113t
